\section{Discussion}

Our findings underscore the importance of both architecture and model size in knowledge grounding. Encoder-decoder architectures, as in Flan-T5, consistently adopt the \Contextual{} answers provided, leading to fewer hallucinations. Smaller models also show better grounding behavior, likely because they rely less on expansive parametric knowledge and are more influenced by explicit context.

For practitioners, these insights suggest that selecting the largest model is not always best. If factual accuracy and adaptability to new evidence are paramount, a Seq2Seq model or a smaller decoder-only model may perform better. Moreover, perplexity can be integrated into retrieval pipelines. When perplexity indicates a misalignment, the system could prompt the retriever for more context or re-check sources, thereby reducing erroneous outputs.

Future work could refine the categorization of answers, using semantic similarity to detect when \Other{} responses are essentially \Parametric{} or \Contextual{} variants. Investigating more subtle contradictions or more complex reasoning tasks would further test LLMs’ ability to integrate external evidence. Additionally, training or fine-tuning models specifically for robust RAG setups might yield even stronger grounding performance.

Overall, this study provides a clearer picture of when and why LLMs defer to provided context, offering practical strategies to enhance reliability in knowledge-intensive settings.
\section{Discussion}
