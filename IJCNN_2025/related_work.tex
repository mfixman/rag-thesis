\section{Related Work}

The success of machine learning models based on transformer architecture \cite{attention_is_all_you_need} has enabled the development of large-scale language models such as GPT-3 \cite{gpt3} and Llama \cite{llama}.
Despite their advancements, factual reliability remains a significant issue.

Recent studies such as Jiang et.\ al \cite{how_can_we_know} highlighted the prevalence of hallucinations across tasks, particularly in factual contexts.
Other studies, such as Ghader et.\ al \cite{can_rag_models_reason}, emphasize the challenge of ensuring accuracy in generated text.

These concerns have prompted a wave of research focused on evaluating and mitigating hallucinations.
Building on this, Cheng et\. al \cite{understanding_the_interplay} systematically explores how parametric and contextual knowledge interact, identifying scenarios where contextual knowledge can degrade performance, even when complementary.

Retrieval-Augmented Generation (RAG) \cite{rag} attempts to improve factual accuracy by integrating external knowledge during inference.
However RAG does not always ensure that language models prioritize the retrieved evidence over their parametric knowledge as evidenced by the research by Yu et\. al and Hsia et\. al \cite{ragged,factual_recall}: even when presented with contradictory context, models often rely on their inherent memory.
Our study builds on these observations, examining this behavior across various model architectures and sizes.

The distinction between parametric knowledge (stored in the model's weights) and contextual knowledge (provided in the input) has been a focal point of several studies.
Qinan Yu et al. \cite{factual_recall} and Chenxi Whitehouse et al. \cite{knowledge_grounding_retrieval_augmented} investigated how factors like training data, architecture, and fine-tuning affect the interplay between these two knowledge sources.
% Their work suggests that Seq2Seq models, such as T5 \cite{t5,flant5}, are generally more effective at using input context compared to decoder-only models, which often struggle to override their internal knowledge.

Through this lens, our work contributes to the understanding of how model architecture, and size shape knowledge grounding in large language models.
