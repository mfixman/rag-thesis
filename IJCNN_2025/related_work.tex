\section{Related Work}
\label{related_work}

% Putting this input here so LaTeX can place it in the right page.
\begin{table*}[t]
	\centering
	\small

	\newcommand{\qtbox}[1]{\parbox{190pt}{#1}}
	\newcommand{\qwbox}[2]{\setlength{\fboxsep}{0pt}\fbox{\colorbox{#1}{\strut \rule{0pt}{8pt}\hspace{1pt}#2\hspace{1pt}}}}
	\newcommand{\qwtext}[2]{\textbf{\color{#1}{#2}}}

	\newcommand{\qwabox}[1]{\qwbox{BurntOrange}{#1}}
	\newcommand{\qwatext}[1]{\qwtext{BurntOrange}{#1}}
	\newcommand{\qwbbox}[1]{\qwbox{ForestGreen}{#1}}
	\newcommand{\qwbtext}[1]{\qwtext{ForestGreen}{#1}}
	\newcommand{\qwcbox}[1]{\qwbox{Thistle}{#1}}
	\newcommand{\qwctext}[1]{\qwtext{Thistle}{#1}}
	\newcommand{\qwdbox}[1]{\qwbox{SkyBlue}{#1}}
	\newcommand{\qwdtext}[1]{\qwtext{SkyBlue}{#1}}
	\newcommand{\qwebox}[1]{\qwbox{Melon}{#1}}
	\newcommand{\qwetext}[1]{\qwtext{Melon}{#1}}

	\newcommand{\qwfbox}[1]{\qwbox{Dandelion}{#1}}
	\newcommand{\qwftext}[1]{\qwtext{Dandelion}{#1}}

	\begin{tabular}{l @{\hspace{-15pt}} r @{\hspace{7pt}} c @{\hspace{50pt}} c @{\hspace{5pt}} l}
		\toprule
			\bfseries Initial Query & \bfseries Parametric Answer && & \bfseries Query with Counterparametric Context \\
		\midrule
			\qtbox{Q: What country is \qwabox{Cairo} in? \\ A: \qwabox{Cairo} is in} & \qwatext{Egypt} & \tikzmark{egyptAnswer} & \tikzmark{usaCtx} & \qtbox{[\qwabox{Cairo} is in \qwbtext{the United States}] \\ Q: What country is \qwabox{Cairo} in? \\ A: \qwabox{Cairo} is in} \\[20pt]
			\qtbox{Q: What country is \qwbbox{New York} in? \\ A: \qwbbox{New York} is in} & \qwbtext{the United States} & \tikzmark{usaAnswer} & \tikzmark{egyptCtx} & \qtbox{[\qwbbox{New York} is in \qwatext{Egypt}] \\ Q: What country is \qwbbox{New York} in? \\ A: \qwbbox{New York} is in} \\[20pt]
			\qtbox{Q: What country is \qwcbox{Bangkok} in? \\ A: \qwcbox{Bangkok} is in} & \qwctext{Thailand} & \tikzmark{thailandAnswer} & \tikzmark{usaCtx2} & \qtbox{[\qwcbox{Bangkok} is in \qwbtext{the United States}] \\ Q: What country is \qwcbox{Bangkok} in? \\ A: \qwcbox{Bangkok} is in} \\[20pt]
			\qtbox{Q: What country is \qwfbox{San Francisco} in? \\ A: \qwfbox{San Francisco} is in} & \qwftext{the United States} & & \tikzmark{thailandCtx} & \qtbox{[\qwfbox{San Francisco} is in \qwctext{Thailand}] \\ Q: What country is \qwfbox{San Francisco} in? \\ A: \qwfbox{San Francisco} is in} \\[20pt]
			\qtbox{Q: What is the date of birth of \qwdbox{Che Guevara}? \\ A: The date of birth of \qwdbox{Che Guevara} is} & \qwdtext{June 14, 1928} & \tikzmark{guevaraAnswer} & \tikzmark{diocletianCtx} & \qtbox{[\qwdbox{Che Guevara} was born in \qwetext{245 CE}] \\ Q: What is the date of birth of \qwdbox{Che Guevara}? \\ A: The date of birth of \qwdbox{Che Guevara} is} \\[20pt]
			\qtbox{Q: What is the date of birth of \qwebox{Emperor Diocletian}? \\ A: The date of birth of \qwebox{Emperor Diocletian} is} & \qwetext{245 CE} & \tikzmark{diocletianAnswer} & \tikzmark{guevaraCtx} & \qtbox{[\qwebox{Emperor Diocletian} was born in \qwdtext{June 14, 1928}] \\ Q: What is the date of birth of \qwebox{Emperor Diocletian}? \\ A: The date of birth of \qwebox{Emperor Diocletian} is} \\[10pt]
		\bottomrule \addlinespace[4pt]
	\end{tabular}

	\begin{tikzpicture}[remember picture, overlay, ->, >=stealth, thick]
		\draw (pic cs:egyptAnswer) +(0pt, 2pt) to (pic cs:egyptCtx);
		\draw (pic cs:usaAnswer) +(0pt,  4pt)  to (pic cs:usaCtx);
		\draw (pic cs:usaAnswer) +(0pt, 0pt) to (pic cs:usaCtx2);
		\draw (pic cs:guevaraAnswer) +(0pt, 2pt) to (pic cs:guevaraCtx);
		\draw (pic cs:diocletianAnswer) +(0pt, 2pt) to (pic cs:diocletianCtx);
		\draw (pic cs:thailandAnswer) +(0pt, 2pt) to (pic cs:thailandCtx);
	\end{tikzpicture}

	\caption{Example of counterparametric context being added to a query on cities. Arrows represent the random shuffling of parametric answers to add to the counterparametric context of a new query with the same base question. We ensure parametric answers aren't shuffled into the context or another question with the same answer, as in the cases with New York and San Francisco.}
	\label{counterparametric_example}
\end{table*}


\textbf{Hallucinations in Large Language Models}
The success of machine learning models based on transformer architecture~\cite{attention_is_all_you_need} has enabled the development of large-scale language models such as GPT-3~\cite{gpt3} and Llama~\cite{llama}.
Despite their advancements, factual reliability remains a significant issue.

Recent studies such as Jiang et al.~\cite{how_can_we_know} highlighted the prevalence of hallucinations across tasks, particularly in factual contexts.
Other studies, such as Ghader et al.~\cite{can_rag_models_reason}, emphasize the challenge of ensuring accuracy in generated text.

These concerns have prompted a wave of research focused on evaluating and mitigating hallucinations.
Building on this, Cheng et al.~\cite{understanding_the_interplay} systematically explores how parametric and contextual knowledge interact, identifying scenarios where contextual knowledge can degrade performance, even when complementary.


\textbf{Parametric and Contextual Knowledge} 
The distinction between parametric knowledge (stored in the model's weights) and contextual knowledge (provided in the input) has been a focal point of several studies.
Yu et al.~\cite{factual_recall} investigated how factors like training data, architecture, and fine-tuning affect the interplay between these two knowledge sources.
% Their work suggests that encoder-decoder models, such as T5~\cite{t5,flant5}, are generally more effective at using input context compared to decoder-only models, which often struggle to override their internal knowledge.

Through this lens, our work contributes to the understanding of how model architecture, and size shape knowledge grounding in large language models.

\textbf{Existing Datasets}: There's a variety of datasets that can be used for analysis of LLMs, such as the Natural Questions dataset~\cite{natural_questions} and the Countries' Capitals dataset~\cite{factual_recall}.
These provide valuable insights, but fall short of the criteria necessary for the experiments being run on this research.

In particular, while the Natural Questions dataset offers a wide range of questions, its lack of systematic categorization hinders counterparametric experiments.
The Countries' Capitals dataset, while well-suited for counterparametric evaluation, is limited in scope.

\textbf{RAG}
Retrieval-Augmented Generation (RAG)~\cite{rag} attempts to improve factual accuracy by integrating external knowledge during inference.
However RAG does not always ensure that language models prioritize the retrieved evidence over their parametric knowledge as evidenced by the research by Yu et al.\ and Hsia et al.~\cite{ragged,factual_recall}: even when presented with contradictory context, models often rely on their inherent memory.
Our study builds on these observations, examining this behavior across various model architectures and sizes.
