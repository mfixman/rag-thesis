\section{Introduction and Objectives}
% This structure is copied from Nikolai Manchev.

\subsection{Problem Background}

In recent years, Large Language Models (LLMs) have become ubiquitous in solving general problems across a wide range of tasks, from text generation to question answering and logic problems.
However, recent research suggests that using these models alone might not be the most effective way to solve problems that are not directly related to text generation \citep{treeofthoughts}.

One approach to improving the performance on knowledge problems for LLMs is Retrieval-Augmented Generation (RAG) \citep{rag}. RAG involves retrieving relevant context related to a query and incorporating it into the model's input, enhancing the model's ability to generate accurate and contextually appropriate responses.

As RAG-enhanced systems become more widespread, studies on the performance of different retrieval systems and their interaction with LLMs have become crucial.
Many explore the performance of these downstream tasks depending on both the retriever and the generator \citep{can_rag_models_reason,fewshotlearners}, examining whether the knowledge is \textit{grounded} in the context.
Retrieval-Augmented models, such as \textsc{Atlas} \citep{atlas_foundational} and \textsc{Retro} \citep{retro}, use this approach to fine-tune a model on both a large body of knowledge and an existing index for context retrieval.

This project aims to understand the performance of various LLMs by measuring their \textit{knowledge grounding} on a dataset consisting of a large variety of questions across a wide range of topics.
We follow the approach by \citeauthor{factual_recall} of running queries with counterfactual context to understand whether a particular answer originates from the model's inherent knowledge (i.e., its training data) or from the provided context (i.e., the context retrieved by RAG).

This thesis builds on this knowledge and improve our understanding of how different LLMs interact with the given context in the problem of question answering.
Specifically, we investigate whether these interactions vary depending on the type of question being answered, contributing to a more nuanced understanding of LLM performance in diverse knowledge domains.

\newpage{}

\subsection{Thesis Questions \& Objectives}

This thesis is structured around three different objectives to deepen our understanding knowledge grounding in large language models.

% We propose the creation of a novel dataset of questions designed for testing LLMs' ability to distinguish between parametric and contextual knowledge.
% This dataset is used to investigate the factors influencing an LLM's choice of answer, and hypothesise that we can use perplexity scores as a predictor of where a particular answer originates.

\subsubsection{Creating a representative dataset of questions}
\label{questions_objective}

The research of this thesis requires a large dataset of questions from a variety of categories to test large language models.
In order to understand knowledge grounding in these models, we require a dataset with the following properties.
\begin{enumerate}
	\item The dataset must contain questions that have short, unambiguous answers.
	\item The questions must cover a large set of topics.
	\item It must allow for the creation of counterfactual answers in the same format as correct ones to test contextual versus inherent knowledge.
\end{enumerate}

The existing literature uses various existing question-and-answer datasets, none of which are useful for this research.\footnotemark{}
\begin{description}
	\item[Natural Questions Dataset] Created by Google Research \citep{natural_questions}, and commonly used in research related to understanding the answers of LLMs in question-and-answer problems \citep{ragged,when_not_to_trust_llms,can_rag_models_reason}.
		While the dataset provides an excellent range of questions and existing literature to compare these results to, the lack of categorisation is an obstacle in our objective to generate counterfactual answers.
	\item[Human-Augmented Dataset] Sometimes used in research related to quality control of large language models \citep{learning_the_difference}.
		However, the high cost associated with this dataset would limit the size of our questions.
	\item[Countries' Capitals Question Dataset] Used in ``Characterizing Mechanisms for Factual Recall in Language Models'' \citep{factual_recall}, this dataset contains a single question about the capital city of certain countries which can be easily transformed to a counterfactual question.
		This format is ideal for the research done in this thesis, but having a single question pattern will not allow a deep dive into the source of each answer in a general question.
\end{description}

Instead of using an existing dataset, this research takes inspiration from the paper by \citeauthor{factual_recall} to create a similar but larger dataset of questions and answers from a wide range of topics, where questions can be grouped by question pattern to ensure that their formats are similar.
This way, we can emulate the approach of that paper of using the answer from a certain question as the counterfactual of another.

This dataset will be used to test the remaining questions of this thesis.
Since it might be useful for future research, it will also be presented as its own result.

\footnotetext[1]{TODO: Maybe this entire subsubsection should go on Section 2 or Section 3.}

\subsubsection{When does a model choose the provided context knowledge over its inherent knowledge?}

Currently, little is understood about the factors and mechanisms that control whether an LLM will generate text respecting either the context or the memorised information.

Previous research found out that, when the context of a query contradicts the ground knowledge of a model, the answer picked depends on the type and size of the model used \citep{factual_recall}.

This thesis extends this research by testing the representative set of questions and counterfactuals described in the previous section with both Seq2Seq and Decoder-only models of various sizes.
We also research the cases when the answer doesn't correspond to either the parametric or contextual knowledge, and why the model chooses a third type of answer when adding counterfactual context.

This thesis also gathers insights from answering this question on different categories and patterns of questions to find out if this depends on what is being asked.

\subsubsection{Can we use the perplexity score of an answer to predict whether it came from inherent or contextual knowledge?}

\Citeauthor{factual_recall} showed that there is a correlation between the probability of a large language model choosing a parametric answer over a counterfactual contextual answer and the amount of times this answer appears in the ground truth data of the model.
This gives us clues on whether the result of a query came from parametric or contextual knowledge if we have access to this ground truth, as is the case in models like Pythia \citep{pythia}.

Unfortunately, most so-called open-source large language models do not give us access to the source data being used to train it and therefore do not allow this kind of analysis.

The \textbf{perplexity} score of answer gives a measure of how ``certain'' a large language model is of its answer \citep{how_can_we_know}.
We hypothesise that we can use this metric to serve as a reliable indicator of whether a particular answer was memorised by the LLM or was derived from the provided context.
