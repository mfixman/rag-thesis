\section{Source Code of the Experiments}
\label{appendixD}

The latest version of the source code, including the input data generated in \cref{creating_dataset}, is available in the mail repository for this thesis \url{https://github.com/mfixman/rag-thesis} and also in its dedicated repository \url{https://github.com/mfixman/knowledge-grounder}.

\lstset{
	language = Python,
	basicstyle = \ttfamily\scriptsize,
	numbers = left,
	numberstyle = \scriptsize,
	numbersep = 5pt,
	backgroundcolor = \color{Gray!10},
	inputencoding = \lstencoding,
	breaklines = true,
	breakatwhitespace = true,
    commentstyle = \color{Gray},
    keywordstyle = \color{blue},
    stringstyle = \color{ForestGreen},
	showspaces = false,
	showstringspaces = false,
    frame = single,
	title = \lstname,
	captionpos = b,
}

\lstinputlisting[caption = \texttt{knowledge\_grounder.py} is the main entry point and contains mostly argument parsing and output printing.]{../knowledge_grounder.py}
\lstinputlisting[caption = \texttt{QuestionAnswerer.py} contains the \texttt{QuestionAnswerer} class which deals with the logic of answering parametric and counterfactual questions from a model]{../QuestionAnswerer.py}
\lstinputlisting[caption = \texttt{Models.py} contains the list of models and includes code that differentiates them.]{../Models.py}
\lstinputlisting[caption = \texttt{Utils.py} contains various useful functions]{../Utils.py}
\lstinputlisting[caption = \texttt{test\_QuestionAnswerer.py} is used to test some of the complicated bits of logic in QuestionAnswerer.]{../test_QuestionAnswerer.py}
