\section{Source Code of the Experiments}

The latest version of the source code, including the input data generated in \cref{creating_dataset}, is available in \url{https://github.com/mfixman/rag-thesis}\footnotemark{}.

\footnotetext{TODO: Move all of this to a new repo.}

\lstset{
	language = Python,
	basicstyle = \ttfamily\scriptsize,
	numbers = left,
	numberstyle = \scriptsize,
	numbesep = 5pt,
	backgroundcolor = \color{Gray!20},
	inputencoding = \lstencoding,
	breaklines = true,
	breakatwhitespace = true,
    commentstyle = \color{Gray},
    keywordstyle = \color{blue},
    stringstyle = \color{ForestGreen},
	showspaces = false,
	showstringspaces = false,
    frame = single,
	title = \lstname,
	captionpos = b,
	titlepos = t,
}

\lstinputlisting[caption = \texttt{knowledge\_grounder.py} is the main entry point and contains mostly argument parsing and output printing.]{../knowledge_grounder.py}
\lstinputlisting[caption = \texttt{QuestionAnswerer.py} contains the \texttt{QuestionAnswerer} class, which deals with the logic of answering parametric and counterfactual questions from a model]{../QuestionAnswerer.py}
\lstinputlisting[caption = \texttt{Models.py} contains the list of models, and includes code that differentiates them.]{../Models.py}
\lstinputlisting[caption = \texttt{Utils.py} contains various usedul functions]{../Utils.py}
