\section{Evaluations, Reflections, and Conclusions}

\subsection{Future Work}

\subsubsection{Better Categorisation of the Answers}

To test whether two answers are equal and to know whether an answer came from parametric or contextual knowledge, the code in this thesis checks for string equality among after removing a few stop simple words such as `the'.

This solution might not be enough, and some answers classified as \textbf{Other} should have been classifier as something else.
\Cref{bad_others} provides some examples of answers where this is the case.

\begin{table}[h]
	\centering
	\scriptsize
	\begin{tabular}{>{\ttfamily}l@{\hspace{20pt}}>{\ttfamily}c@{\hspace{1pt}}>{\ttfamily}c@{\hspace{1pt}}c@{\hspace{1pt}}c}
		\toprule
			\bfseries \rmfamily Query & \bfseries \rmfamily Parametric Answer & \bfseries \rmfamily Query Answer & \bfseries Comparison & \bfseries Expected \\
		\midrule
			\parbox{100pt}{[Context: The primary leader associated with The Construction of Hadrian's Wall was Napoleon Bonaparte] \\ Q: Who was the primary leader associated with The Construction of Hadrian's Wall? \\ A: The primary leader associated with The Construction of Hadrian's Wall was} &
			Emperor Hadrian &
			\parbox[c][][c]{75pt}{\centering Hadrian, \\ the Roman Emperor} &
			\bfseries \textcolor{MidnightBlue}{Other} &
			\bfseries \textcolor{ForestGreen}{Parametric} \vspace{4pt} \\
		\midrule
			\parbox{100pt}{[Context: Che Guevara was born in Kensington, London, England] \\ Q: In what city was Che Guevara born? \\ A: Che Guevara was born in} &
			Rosario, Argentina &
			London &
			\bfseries \textcolor{MidnightBlue}{Other} &
			\bfseries \textcolor{Maroon}{Contextual} \\
		\bottomrule
	\end{tabular}
	\caption{Example of incorrectly-categorised answers. These were categorised as "Other", since their answer strings are different from both parametric and contextual answers. However, a closer look reveals that this is just either answer with a slight formatting difference.}
	\label{bad_others}
\end{table}

A more complete solution might include using another LLM to compare whether two answers are truly equal.

\subsubsection{Knowledge Grounding in Retrieval-Augmented LMs}

This thesis was originally based on a preprint, ``Knowledge Grounding in Retrieval-Augmented LMs: An Empirical Study'' \citep{knowledge_grounding_retrieval_augmented}, and contains work towards understanding how large language models retrieve data which can ultimately help prevent hallucinations.

We plan to continue this work and complete the paper created by the preprint by running the methods outlined on this thesis on retrieval-augmented LMs such as \textsc{Atlas} \citep{atlas_foundational} and \textsc{Retro} \citep{retro} and creating a full evaluation framework that specifically focuses on their grounding.
A well-grounded model should demonstrate the capability to adapt its generation based on the provided context, specially in cases like the ones experimented in this thesis when the context contradicts the model's parametric memorisation.

\subsubsection{Further Memory Locator Prediction}

The results of \cref{results_perplexity_score} show a clear difference in perplexity value between answers that come from the parametric memory of a model and those that come from a context.

This could be used to create a predictor where, given a certain answered query, it could give you a probability of the source the model used for this answer by using the perplexity of the answer and comparing against the distribution of perplexities for this model on similar questions.

In RAG-enhanced models, where the RAG context might contradict the parametric knowledge of a model, this might prevent hallucinations.

\subsubsection{Fine-tuning a LLM for a RAG Context}

Existing retrieval-augmented LMs, such as \textsc{Atlas} and \textsc{Retro}, are trained on existing models along with an index.
In the fast-moving world of large language models, this might not be ideal: the generator part of models is based on T5, a model created in 2019.
Meanwhile, between the time I started writing this thesis and this moment Meta launched a new Llama model.

The current dataset and experiments might be useful for being able to fine-tune a modern model to prefer the context generated by RAG when it contradicts its parametric knowledge.
This might improve retrieval-augmented models, and make it easier to use them with newer models.
