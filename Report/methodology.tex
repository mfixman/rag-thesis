\section{Methodology}

\newcommand{\cats}{7}
\newcommand{\baseqs}{68}
\newcommand{\things}{312}
\newcommand{\qs}{3840}

\subsection{Source Data Preparation}

Our source data is prepared by extending the ideas presented by Yu et al\cite{factual_recall}.
Instead of using one simple question, our approach consists of separating this data into \cats{} categories, where each category has a set of base questions and another set of objects that are paired together and presented to our models.

This work contains \cats{} categories in the configuration shown by \cref{categories_numbers}, for a total of \qs{} questions.
The full list of questions can be found in \cref{appendixA}.

\begin{table}[h]
	\centering
	\footnotesize
	\begin{tabular}{l | r r r}
		\toprule
		Category & Questions & Objects & Total \\
		\midrule
			Person           & 14 &  47 &  658 \\
			City             & 14 &  60 &  840 \\
			Principle        & 10 &  30 &  300 \\
			Element          & 10 &  35 &  350 \\
			Book             & 10 &  45 &  450 \\
			Painting         & 14 &  39 &  546 \\
			Historical Event & 6  &  56 &  336 \\
		\midrule                    
			Total            & \baseqs{} & \things{} & \qs{} \\
		\bottomrule
	\end{tabular}
	\caption{The amount of questions for each category. The full list of questions can be found in \cref{appendixA}. This is still a work in progress and I expect to add more questions.}
	\label{categories_numbers}
\end{table}

For prompt engineering each one of these \baseqs{} questions 

Each one of these \qs{} questions is generated by adding the object 
